\thispagestyle{empty}
\newcommand{\form}[1]{\scalebox{1.087}{\boldmath{#1}}}
\sffamily
%
\begin{textblock}{191}(-24,-11)
\colorbox{blue}{\hspace{110mm}\ \parbox[c][18truemm]{81mm}{\textcolor{white}{FACULTY OF ECONOMICS \& BUSINESS}}}
\end{textblock}
%
\begin{textblock}{70}(-18,-19)
\textblockcolour{}
\includegraphics*[height=30truemm]{Cover/KULEUVEN_FEB_Logo}
\end{textblock}
%
\begin{textblock}{160}(-6,63)
\textblockcolour{}
\vspace{-\parskip}
\flushleft
\fontsize{40}{42}\selectfont \textcolor{bluetitle}{\form{Master Thesis}}\\[1.5mm]
\fontsize{20}{22}\selectfont \form{Visualisation of IoT Business Processes}
\end{textblock}
%
%\begin{textblock}{79}(50,103)
%\textblockcolour{}
%\vspace{-\parskip}
%\flushleft
%\fbox{\parbox{79mm}{De achtergrond kan wit blijven of je kan een afbeelding invoegen (maximum hoogte 10 cm, breedtevariabel, denk aan auteursrechten\ldots). GEEN logo's (je kan binnenin de masterproef logo's gebruiken, maar niet op de voor- %of achterpagina). \textit{Verwijder deze tekstkader.}}}
%\end{textblock}
%
\begin{textblock}{160}(8,153)
\textblockcolour{}
\vspace{-\parskip}
\flushright
\end{textblock}
%
\begin{textblock}{70}(-6,191)
\textblockcolour{}
\vspace{-\parskip}
\flushleft
%Promotor: Prof. A. Xyz\\[-2pt]
%\textcolor{blueaff}{Affiliatie \textsl{(facultatief)}}\\[5pt]
%Co-promotor: \textsl{(facultatief)}\\[-2pt]
%\textcolor{blueaff}{Affiliatie \textsl{(facultatief)}}\\[5pt]
%\textcolor{blueaff}{Affiliatie \textsl{(facultatief)}}\\
\end{textblock}
%
\begin{textblock}{160}(8,191)
\textblockcolour{}
\vspace{-\parskip}
\flushright
%Proefschrift ingediend tot het\\[4.5pt]
%behalen van de graad van\\[4.5pt]
%Bachelor of Science in Xxx\\
\end{textblock}
%
\begin{textblock}{160}(8,205)
\textblockcolour{}
\vspace{-\parskip}
\flushright
\fontsize{14}{16}\selectfont \textbf{Michael Vet}\\
\fontsize{14}{16}\selectfont \textsl{r0829545}\\
\vspace{5mm}
Prof. Dr. Jochen De Weerdt\\
Mr. Yannis Bertrand\\
Academic year 2021-2022
\end{textblock}
%
\begin{textblock}{191}(-24,248)
{\color{blueline}\rule{550pt}{5.5pt}}
\end{textblock}
%
\vfill
