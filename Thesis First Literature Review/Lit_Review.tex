%\section{Literature Review}

\begin{multicols}{2}

\subsection{IoT-driven Event Log Generation}
The Internet of Things enables the digitization of the physical world through interconnected devices, thereby providing access to vast amounts of data that can be used to develop digital services in several application domains including Business Process Management (BPM). The data generated by sensing devices and smart objects allows for the continuous monitoring of the IoT devices and their surroundings, providing new opportunities for analysis and optimization of the processes performed in IoT environments, e.g., through process mining approaches \cite{seiger_towards_2020}.

Meroni et al. exploit in \cite{meroni_multi-party_2018} the Internet of Things (IoT) Paradigm by equipping physical objects with sensing hardware and software, turning them into smart objects. %The monitoring of multi-party business processes is a challenging activity: needs for coordination among the involved BPMSs, limited visibility on the whole process by different parties, differences in monitoring artifacts and control flows are some of the aspects that traditional monitoring solutions are not able to cope with. 
As each smart object is referring to one of the physical artifacts involved, it will keep the owner of the artifact informed about the progression of its state and how the process is evolving.


\subsection{Concept Drift}
The phenomenon in which the statistical properties of a target domain change over time is considered concept drift \cite{desai_issue_2021}.
%In predictive analytics and machine learning, concept drift means that the statistical properties of the target variable, which the model is trying to predict, change over time in unforeseen ways \cite{zheng_detecting_2017}.
Processes may change due to periodic/seasonal changes or due to changing conditions. Such changes impact processes and it is vital to detect and analyze them \cite{van_der_aalst_conformance_2016}.

There are three challenges when dealing with concept drift \cite{bose_handling_2011}:
\begin{enumerate}
    \item Change (Point) Detection: Detect that a process has changed. If so, identify the time periods at which changes have taken place.
    \item Change Localization and Characterization: Once a point of change has been identified, the next step is to characterize the nature of change, and identify the region(s) of change (localization) in a process.
    \item Unravel Process Evolution: Relate the previous discoveries to unravel the evolution of a process. This should lead to the discovery of the change process.
\end{enumerate}


C. Zheng et al. suggest in \cite{zheng_detecting_2017} TPCDD (Tsinghua Process Concept Drift Detection) to extract relations from each trace and transform the event log into a relation matrix. Then, for each relation we observe its variation trend and detect candidate change points. Finally all candidate change points are clustered to get a final result.  

\subsection{Visualization of Compliance Violation}
 BPMN-Q queries can be used to express compliance rules regarding execution ordering of activities. For each query a set of anti pattern queries is automatically derived and checked against process models as well. When a violation occurs (an anti pattern finds a match), the violating part of the process is shown to the user \cite{awad_visualization_2010}.

%Facing concept drifts, process analysts require methods and tools that are capable of detecting change points from event logs, helping them to analyze the evolution of processes. Mainstream is change-point detection: the goal is to detect changes in the generating distributions of the time-series. (performance not satisfactory)

%Concept drift, also known as process drift in a PM context, refers to business changes over time, which can be detected from various perspectives, such as control-flow, data-flow, resource and time. Detecting such changes can provide insight into business circumstances, generate early-stage change warnings and highlight the opportunity for model refinement.

\end{multicols}